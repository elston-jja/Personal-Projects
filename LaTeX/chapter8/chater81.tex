\documentclass{article}
\usepackage{mhchem, hhline, adjustbox}
\begin{document}


\setcounter{section}{7}
\section{The nature of Acids and Bases \tiny{ || Elston Almeida}}

\subsection{Arrhenius}


\noindent
\begin{center}

  Acid \ce{H+} ions \hspace{5pc} Bases \ce{OH-} ions

\begin{align*}
  \ce{ HCl(aq) &->[water] H+(aq) + Cl-(aq)}\\
  \ce{ NaOH(s) &->[water] OH-(aq) + Na+(aq)}
\end{align*}

\end{center}

\textbf{Br{\o}nsted Theory}
\\
\begin{center}
  Acid is a hydrogen donor.
  Base is a hydrogen acceptor.
\end{center}
  \subsection{Br{\o}nsted lowry acids}
\begin{center}
  \begin{align*}
  \ce{HCl(aq) + H2O(l) <=> Cl-(aq) + H3O+(aq)}
  \end{align*}
\end{center}

Water can be both acids and bases depending on the reaction. In the reaction above, it acts as an base to accept the hydrogen ion. (HCl donates the H ion to the water)

\vspace{-1pc}
\begin{center}
  
  \begin{align*}
    \ce{ NH3(aq) + H2O(l) <=> NH4+(aq) + OH-(aq)}
  \end{align*}

\end{center}

NH3 is an base, H2O acts as an acid, NH4 is the conjugate acid, OH is the conjugate base\\

Conjugate acid is when the substance that forms then a base accepts a hydrogen ion.
Conjugate base is when the substance that forms when an acid loses a hydrogen ion.\\

Amphiprotic is a substance that can donate and accept a hydrogen ion. Water is a Amphiprotic as you can get Hydronium, and Hydroxide.\\

$K_a$ value is the eq constant for the ionization of an acid (called the acid dissociation constant)
\\\\
General Equation:
\begin{center}
\begin{equation}
\ce{K_a = \frac{[H3O^+(aq)][A^-(aq)]}{[HA(aq)]}}
\end{equation}
\end{center}
\noindent
\newpage
Example:
\begin{center}
  \ce{HC2H3O2(aq) + H2O(l) <=> H3O+(aq) + C2H3O2-(aq)}
  \begin{equation*}
    \ce {K_a = \frac{[H3O^+(aq)][C2H3O2^-(aq)]}{[HC2H3O2(aq)]}} 
  \end{equation*}
\end{center}
HW492 \#1, 493 \#1.

\subsection{Strong and Weak Acids}

\begin{center}

  A strong acid ionizes almost completely in water\\
  A weak acid is one that only partially ionizes in water

  \begin{align*}
    \ce{HCl(aq) + H_2O(l) <=> H_3O+(aq) + Cl-(aq)}
  \end{align*}
  
\end{center}

\vspace{5mm}
\newcommand{\br}{\hhline{||-||-||-||}}
\begin{adjustbox}{center}
\def\arraystretch{1.3}
\begin{tabular}{||c||c||c||}
  \hhline{|t:=:=:=:t|}
  \textbf{Property}& \textbf{Strong Acid} & \textbf{Weak Acid}\\
  \hhline{||=||=||=||}
  Value of acid ionization constant, $K_a$& $K_a$ is large & $K_a$ is small\\
  \br
  Position of the ionization equilibrium & far to the right & far to the left\\
  \br
  Equilibrium concentraton of \ce{H+ (aq)}  & \ce{[H+ (aq)]_{eq} \approx[HA (aq)]_i} & \ce{[H+ (aq)]_{eq} $<<$ [HA (eq)]_i}\\
  compared with the original concentration& Equal. pH $\approx$ Initial pH & Equal. pH $<<$ Initial pH\\
   \hhline{|b:=:=:=:b|}
\end{tabular}
\end{adjustbox}

\subsection{Stong and Weak Bases}
\begin{center}
Strong base dissociates completely in water\\
Weak base partially dissociates in water\\
\end{center}
The base ionization constant(Kb) is the base equilibrium constant for the ionization of a base(it is also called the base dissociation constant)\\

\begin{center}
  \begin{equation}
    K_b =\ce{\frac{[BH^+(aq)][OH^-(aq)]}{[B(aq)]}}
  \end{equation}
  \end{center}
Example:\\
\begin{center}
\ce{NH3(aq) + H2O(l) <=> NH4+(aq) + OH-(aq)}
\end{center}
\begin{align*}
K_b = \frac{\ce{[OH-(aq)][NH4+(aq)]}}{\ce{[NH3(aq)]}}
\end{align*}

\vspace{4mm}
Chart for $K_b$ for weak acids: pg 727 
\newpage

The autoionization of water is the transfer of a hydrogen ion from one water molecule to another.

\begin{center}
\ce{2H2O(l) <=> H3O+(aq) + OH-(aq)}
\end{center}
\begin{center}
  \begin{align*}
    \ce{K_w&=[H3O+(aq)][OH^-(aq)]}\\
    \ce{K_w&=[1.0*10^-7][1.0*10^-7]}\\
    \ce{K_w&=1.0*10^-14}
  \end{align*}
\end{center}

\begin{center}
$K_w$ is always $1.0*10^{-14}$ at SATP\\
\end{center}
  
\vspace{1mm}

\begin{align*}
[H] &= [OH] && \text{\textbf  {Neutral} solution}\\
[H] &> [OH] && \text{\textbf {Acidic} solution}\\
[H] &< [OH] && \text{\textbf  {Basic} solution\hspace{3.3mm}}  \\
\end{align*}
\\
Example to find the \ce{[H3O^-(aq)]}

\begin{center}
  \begin{align*}
    K_w&=\ce{[H3O+(aq)][OH^-(aq)]}\\
    \ce{[H3O+(aq)]}&=\frac{K_w}{\ce{[OH^-(aq)]}}
  \end{align*}
\end{center}
\begin{center}
  \begin{equation*}
   K_w=K_a\cdot{K_b}
    \end{equation*}
\end{center} 
\begin{center}
\begin{align*}
pH&=-log[H]&&
[H]=10^{-pH}\\
 pOH&=-log[OH]&&
 [OH]=10^{-pH}\\
 \vspace{4mm}
 14&=pH+pOH &&
 pK_w = pH + pOH 
\end{align*}
\end{center}

pH meter is an electronic devide that measures the acidity of a solution and displays the result as a pH value.\\

An acid base indicator is a substance that changes color specific to the pH range\\

pg 495-509

pg 502 \#1,2;
pg 505 \#1-3;
pg 508 \#1-4;
pg 509 \#1-10

\newpage
 
\subsection{Calculations involving acidic solutions}

Since strong acids almost completely ionize in water, we can assume that the concentration of hydrogen ions is equal to the concentration of the acid.\\

Ex a solution of hydrochloric acid has a concentration od 0.1M. Calculate:
\vspace{4mm}
\\
\begin{center}
\ce{HCl(aq) + H2O(l) <=> H3O-(aq) + Cl-(aq)}
\end{center}
\ce{[H+]} = 0.1M \\
\ce{[OH^-]} = 1*10 e -13\\
\ce{pH} = 1\\
\ce{pOH} = 13\\

Precentage ionization is the percentage of a solute that ionizes when it dissolves in a solvent.\\

%% \begin{equation}
%%   $ionization = \frac{ionized concentration acid}}{initial conecntration acid}$\\
%% \end{equation}
 
Calculate the ka hydrofluoric acid HF, if a 0.100M solution at equilibrium has a percentage ionization.

\ce{HF(aq) + H20(l) <=> H3O+(aq) + F-(aq)}

\end{document}
