\documentclass[12pt]{article}
\usepackage[version=4]{mhchem}
\usepackage{adjustbox}
\usepackage{enumitem}
\begin{document}

\title{ \vspace{-2cm} \textbf {Le Chatelier's principle}\\ Lab 2}
\author{Matthew S. \and Elston A.}

\maketitle

\section{Prelab}

\textbf 1. When a solution reaches equlibrium, all of the macroscopic properties remain constant, this means that color of a solution at equlibrium does not change. So the solution that is not changing color is at equlibrium. \\

\noindent
\textbf 2a. When the potassium bromide was added to the solution, it ionized into potassium ions and bromide ions. These bromide ions reacted with the dissoved hydrated copper ions to produce a higher concentration of copper bromide ions.\\

\noindent
 \textbf 2b. The solution changed color due to the higher concentration of copper bromide. Since copper bromide ions are blueish, it changed the overall apperarence of the test tube.\\

 \noindent
 \textbf 2c. The reaction to produce copper bromide ions is endothermic, therefore, it will absorb the thermal energy from the environment, leaving the test tube cold to the touch.\\

\noindent
\textbf 3. The unknown checical was an acid which caused a newtralization reaction with magnesium hydroxide producing water and aqueous salts. Therefore leaving the cloudy test tube clear.\\

\noindent
\textbf 4. The removal of methanol would induce a complete reaction as there would be no product to preform a reverse reaction. In other words, it would force the equlibrium as far right as to use up all carbon monoxide and hydrogen to make the methanol.
 
\newpage

\section{Experiment}


\fontsize{9pt}{12pt}
Note: ``\textbf s'' represents stress. The stress indicates how the system was affected, and all other values is how the system responded to the stress.
\fontsize{12pt}{12pt}


\begin{center}
  \ce{ CoCl4^2- ${(blue)}$ + 6H_2O <=> Co (H2O)6^2+ ${(red)}$ + 4Cl^- + Heat }
\end{center}
\begin{adjustbox}{max width=50in,center}
  \begin{tabular}{|c|c|c|c|c|c|c|c| } 
    \hline
    Addition&$CoCl_4^{2-}$&$H_2O$&shift&$Co(H_2O)^{2+}$&$Cl^{-}$&Heat&Color \\
    \hline
    Heat&increase&increase&left&decrease&decrease&decrease \textbf s $\uparrow$&blue\\
    \hline
    Cool&decreases&decreases&right&increase&increase&increase \textbf s $\downarrow$&red\\
    \hline
    $AgNO_{3(aq)}$&increase&increase&left&increase &decrease \textbf s $\uparrow$ & decrease& blue\\
    \hline
    $H_2O_{(l)}$&decrease&decrease \textbf s $\uparrow$ & right& increase& increase & increase& red\\
    \hline
  \end{tabular}
\end{adjustbox}
\vspace{0cm}

\begin{center}
  \ce{  Heat + NH_4^+ + OH^- <=> NH_3 + H_2O}
\end{center}
\begin{adjustbox}{max width=50in,center}
  \begin{tabular}{|c|c|c|c|c|c|c|c|}
    \hline
    Addition&Heat&$NH_4^{+}$&$OH^{-}$&shift&$NH_3$&$H_2O$&Color\\
    \hline
    $NH_4Cl_{(aq)}$&decrease&decrease \textbf s $\uparrow$& decrease& right & increased& increased& colorless\\
    \hline
    $HCl_{(aq)}$&increase&increase&increase \textbf s $\downarrow$& left& decrease & decrease \textbf s $\uparrow$ & colorless\\ 
    \hline
    $NaOH_{(aq)}$&decrease&decrease&decrease \textbf s $\uparrow$ & right&increase&increase&purple\\
    \hline
  \end{tabular}
\end{adjustbox}

\vspace{0cm}
\begin{center}
  \ce {Fe^{+3}$(pale yellow)$ + SCN^- <=> FeSCN^{+2} $(red)$ + Heat}
\end{center}
\vspace{0cm}
\begin{adjustbox}{max width=50in,center}
  \begin{tabular}{|c|c|c|c|c|c|c|}
    \hline 
    Addition&$Fe^{+3}$&$SCN^-$&shift&$FeSCN^{+2}$&Heat&Color\\
    \hline
    $KSCN$&decrease&decrease \textbf s $\uparrow$&right&increase&increase&red\\
    \hline
    $Fe(NO_3)_3$&decrease \textbf s $\uparrow$& decrease&right&increase&increase&red\\
    \hline
    Heat&increase&increase&left&decrease&decrease \textbf s $\uparrow$&yellow\\
    \hline
    $Na_2HPO_4$&increase \textbf s $\downarrow$&increase&left&decrease&decrease&yellow\\
    \hline
    \end{tabular}
\end{adjustbox}
\vspace{0cm}

\begin{center}

  \ce{ 2 CrO_4^{-2} $(yellow)$ + 3 H^+ <=>  Cr_2O_7^{-2} $(red)$ + H_2O }
\end{center}
\vspace{0cm}
\begin{adjustbox}{max width=50in,center}
  \begin{tabular}{|c|c|c|c|c|c|c|}
    \hline
    Addition & $CrO_4^{-2}$ & $H^+$ & shift & $Cr_2O_7^{-2}$&$H_2O$& Color\\
    \hline
    $HCl_{(aq)}$ &decrease&decrease \textbf s $\uparrow$&right&increase&increase&red\\
    \hline
    $NaOH$&increase&increase \textbf s $\downarrow$&left&decrease&decrease&yellow\\
    \hline
    
  \end{tabular}
  
\end{adjustbox}
\newpage

\begin{center}

  \ce{2 NO_2 $(brown)$ <=> N_2O_4 $(colorless)$ + Heat}
  
\end{center}
\vspace{0cm} 
\begin{center}
  \begin{tabular}{|c|c|c|c|c|c|}
    \hline
    Addition&$NO_2$&shift&$N_2O_4$&Heat & Color\\
    \hline
    Cool&decrease&right&increase&increase \textbf s $\downarrow$& colorless\\
    \hline
    Heat&increase&left&decrease&decrease \textbf s $\uparrow$& brown\\
    \hline
    
  \end{tabular}
\end{center}
\vspace{0cm}
\begin{center}
\ce{ CuSO_4 * 5 H_2O + Heat <=> CuSO_4 + 5 H_2O}
\end{center}
\vspace{0cm}
\begin{center}
\begin{tabular}{|c|c|c|c|c|c|c|c|}
  \hline
  Addition& $CuSO_4 \cdot 5H_2O$ & Heat& shift & $CuSO_4$&$5H_2O$ & Color\\

  \hline

  Heat&decrease&decrease \textbf s $\uparrow$&right&increase&increase&brown\\

  \hline
  
\end{tabular}

\end{center}

\section{Postlab}


\noindent
\textbf 1. The addition of heat in the ammounium reaction would cause the equilibrium to shift to the right.\\

\noindent
\textbf 2. Ammonia was added to the copper solution to produce the dark blue solution of copper ammonia ions. In order to get the light blue solution, you would be required to add an excess of copper ammonia in order to force equlibrium to the left as to produce more copper ions which are light blue in color.\\

\noindent
\textbf 3. The equilibrium would be shifted to the right, as to produce more barium sulfate. This would mean that solution produced would be less toxic to be used for the ingestion. \\

\noindent
\textbf 4. Removing the patient to an area of fresh air would decrease the concentration of carbon monoxide in the environment, this would shift the equilibrium to the left displacing the carbon monoxide attached to the hemoglobin cells. Suppling the patient with pure oxygen would have a stronger effect with the same result. \\

\end{document}
