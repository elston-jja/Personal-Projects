\documentclass{article}
\usepackage{mhchem, hhline, adjustbox, enumitem}
\begin{document}


\setcounter{section}{7}
%% \section{The nature of Acids and Bases \tiny{ || Elston Almeida}}

%% \subsection{Arrhenius}


%% \noindent
%% \begin{center}

%%   Acid \ce{H+} ions \hspace{5pc} Bases \ce{OH-} ions

%% \begin{align*}
%%   \ce{ HCl(aq) &->[water] H+(aq) + Cl-(aq)}\\
%%   \ce{ NaOH(s) &->[water] OH-(aq) + Na+(aq)}
%% \end{align*}

%% \end{center}

%% \textbf{Br{\o}nsted Theory}
%% \\
%% \begin{center}
%%   Acid is a hydrogen donor.
%%   Base is a hydrogen acceptor.
%% \end{center}
%%   \subsection{Br{\o}nsted lowry acids}
%% \begin{center}
%%   \begin{align*}
%%   \ce{HCl(aq) + H2O(l) <=> Cl-(aq) + H3O+(aq)}
%%   \end{align*}
%% \end{center}

%% Water can be both acids and bases depending on the reaction. In the reaction above, it acts as an base to accept the hydrogen ion. (HCl donates the H ion to the water)

%% \vspace{-1pc}
%% \begin{center}
  
%%   \begin{align*}
%%     \ce{ NH3(aq) + H2O(l) <=> NH4+(aq) + OH-(aq)}
%%   \end{align*}

%% \end{center}

%% NH3 is an base, H2O acts as an acid, NH4 is the conjugate acid, OH is the conjugate base\\

%% Conjugate acid is when the substance that forms then a base accepts a hydrogen ion.
%% Conjugate base is when the substance that forms when an acid loses a hydrogen ion.\\

%% Amphiprotic is a substance that can donate and accept a hydrogen ion. Water is a Amphiprotic as you can get Hydronium, and Hydroxide.\\

%% $K_a$ value is the eq constant for the ionization of an acid (called the acid dissociation constant)
%% \\\\
%% General Equation:
%% \begin{center}
%% \begin{equation}
%% \ce{K_a = \frac{[H3O^+(aq)][A^-(aq)]}{[HA(aq)]}}
%% \end{equation}
%% \end{center}
%% \noindent
%% \newpage
%% Example:
%% \begin{center}
%%   \ce{HC2H3O2(aq) + H2O(l) <=> H3O+(aq) + C2H3O2-(aq)}
%%   \begin{equation*}
%%     \ce {K_a = \frac{[H3O^+(aq)][C2H3O2^-(aq)]}{[HC2H3O2(aq)]}} 
%%   \end{equation*}
%% \end{center}
%% HW492 \#1, 493 \#1.

%% \subsection{Strong and Weak Acids}

%% \begin{center}

%%   A strong acid ionizes almost completely in water\\
%%   A weak acid is one that only partially ionizes in water

%%   \begin{align*}
%%     \ce{HCl(aq) + H_2O(l) <=> H_3O+(aq) + Cl-(aq)}
%%   \end{align*}
  
%% \end{center}

%% \vspace{5mm}
%% \newcommand{\br}{\hhline{||-||-||-||}}
%% \begin{adjustbox}{center}
%% \def\arraystretch{1.3}
%% \begin{tabular}{||c||c||c||}
%%   \hhline{|t:=:=:=:t|}
%%   \textbf{Property}& \textbf{Strong Acid} & \textbf{Weak Acid}\\
%%   \hhline{||=||=||=||}
%%   Value of acid ionization constant, $K_a$& $K_a$ is large & $K_a$ is small\\
%%   \br
%%   Position of the ionization equilibrium & far to the right & far to the left\\
%%   \br
%%   Equilibrium concentraton of \ce{H+ (aq)}  & \ce{[H+ (aq)]_{eq} \approx[HA (aq)]_i} & \ce{[H+ (aq)]_{eq} $<<$ [HA (eq)]_i}\\
%%   compared with the original concentration& Equal. pH $\approx$ Initial pH & Equal. pH $<<$ Initial pH\\
%%    \hhline{|b:=:=:=:b|}
%% \end{tabular}
%% \end{adjustbox}

%% \subsection{Stong and Weak Bases}
%% \begin{center}
%% Strong base dissociates completely in water\\
%% Weak base partially dissociates in water\\
%% \end{center}
%% The base ionization constant(Kb) is the base equilibrium constant for the ionization of a base(it is also called the base dissociation constant)\\

%% \begin{center}
%%   \begin{equation}
%%     K_b =\ce{\frac{[BH^+(aq)][OH^-(aq)]}{[B(aq)]}}
%%   \end{equation}
%%   \end{center}
%% Example:\\
%% \begin{center}
%% \ce{NH3(aq) + H2O(l) <=> NH4+(aq) + OH-(aq)}
%% \end{center}
%% \begin{align*}
%% K_b = \frac{\ce{[OH-(aq)][NH4+(aq)]}}{\ce{[NH3(aq)]}}
%% \end{align*}

%% \vspace{4mm}
%% Chart for $K_b$ for weak acids: pg 727 
%% \newpage

%% The autoionization of water is the transfer of a hydrogen ion from one water molecule to another.

%% \begin{center}
%% \ce{2H2O(l) <=> H3O+(aq) + OH-(aq)}
%% \end{center}
%% \begin{center}
%%   \begin{align*}
%%     \ce{K_w&=[H3O+(aq)][OH^-(aq)]}\\
%%     \ce{K_w&=[1.0*10^-7][1.0*10^-7]}\\
%%     \ce{K_w&=1.0*10^-14}
%%   \end{align*}
%% \end{center}

%% \begin{center}
%% $K_w$ is always $1.0*10^{-14}$ at SATP\\
%% \end{center}
  
%% \vspace{1mm}

%% \begin{align*}
%% [H] &= [OH] && \text{\textbf  {Neutral} solution}\\
%% [H] &> [OH] && \text{\textbf {Acidic} solution}\\
%% [H] &< [OH] && \text{\textbf  {Basic} solution\hspace{3.3mm}}  \\
%% \end{align*}
%% \\
%% Example to find the \ce{[H3O^-(aq)]}

%% \begin{center}
%%   \begin{align*}
%%     K_w&=\ce{[H3O+(aq)][OH^-(aq)]}\\
%%     \ce{[H3O+(aq)]}&=\frac{K_w}{\ce{[OH^-(aq)]}}
%%   \end{align*}
%% \end{center}
%% \begin{center}
%%   \begin{equation*}
%%    K_w=K_a\cdot{K_b}
%%     \end{equation*}
%% \end{center} 
%% \begin{center}
%% \begin{align*}
%% pH&=-log[H]&&
%% [H]=10^{-pH}\\
%%  pOH&=-log[OH]&&
%%  [OH]=10^{-pH}\\
%%  \vspace{4mm}
%%  14&=pH+pOH &&
%%  pK_w = pH + pOH 
%% \end{align*}
%% \end{center}

%% pH meter is an electronic devide that measures the acidity of a solution and displays the result as a pH value.\\

%% An acid base indicator is a substance that changes color specific to the pH range\\

%% pg 495-509

%% pg 502 \#1,2;
%% pg 505 \#1-3;
%% pg 508 \#1-4;
%% pg 509 \#1-10

%% \newpage
 
\subsection{Calculations Involving Acidic Solutions}

Since strong acids almost completely ionize in water, we can assume that the concentration of hydrogen ions is equal to the concentration of the acid.\\

\noindent
Ex: A solution of hydrochloric acid has a concentration of 0.1M.
\\
\\Calculate:
\begin{center}
\ce{HCl(aq) + H2O(l) <=> H3O-(aq) + Cl-(aq)}
\end{center}
\begin{align*}
\ce{[H+]} &= 0.1M \\
\ce{[OH^-]} &= 1\cdot10^{13}M\\
\ce{pH} &= 1\\
\ce{pOH} &= 13\\
\end{align*}
\noindent
Precentage ionization is the percentage of a solute that ionizes when it dissolves in a solvent.\\

\begin{equation*}  
  \text{\% ionization} = \frac{\text{[ Ionized Acid ]}}{\text{[ Initial Acid ]}}\\
\end{equation*}
 
\subsection{Monoprotic and Polyprotic acids}

Monoprotic acid is an acid that possess only one ionizable hydrogen acid.\\

\noindent
Polyprotic acid is an acid that possess more than one ionizable.\\

\noindent
\ce{K_{a1}} is larger than the other \ce{K_{an}} values:
\begin{center}
\ce{K_{a1} $>$ K_{a2} $>$ K_{a3}}\\
\end{center}

\noindent
If the 5 percent rule is does not work the pH is zero\\

\subsection{Calculations involving basic solutions}

\begin{center}
  \ce{Ca(OH)2(s) <=> Ca2+ + 2OH-}
\end{center}

\noindent
Metal oxides dissolve in water to produce a basic solution\\

\noindent
Non-metallic oxides dissolve in water to produce acidic 

\subsection{Acid-Base Titration}
\vspace{2mm}
 \begin{description}
 \item[$\bullet$]\noindent Titration is to determine the ph of a solution by neutralization.

\item[$\bullet$]\noindent Titrations are used the determine the concentration of an acid or base.

\item[$\bullet$]\noindent  The equivilance point is the point of the titration when the acid and base completely react with each other.

\item[$\bullet$]\noindent If you know the columes of both solution at the equivialnce point, and the concentrations of one of them, you can calcualte the unknown concentration.
 \end{description}
 \subsection{Buffers and the Common Ion Effect}
 \vspace{2mm}
 The Solubility of a partially soluble salt is decreased with the addition of a common ion.\\

 \noindent
 A buffered solution consists of a mixture of a weak acid and its conjugate base\ce{X-}:

 \begin{center}
   \ce{HX (aq) <=> H+ (aq) + X- (aq)}
 \end{center}

 \begin{align*}
   \ce{K_a &= \frac{[H+][X-]}{[HX]}}
 \end{align*}
\\
\textbf{Buffer Capacity and pH}
\\
\begin{description}
\item[$\bullet$]\noindent Buffered capacity is the amount of acid or base neutralized by the buffer before there is a signifigant change in pH.
\item[$\bullet$]\noindent Buffer capacity depends of the composition of the buffer.
\item[$\bullet$]\noindent The greater the amounts of the conjugate acid-base pair, the greater the buffer capacity.
\item[$\bullet$]\noindent The pH of the buffer depends on \ce{K_a}.
\end{description}


\end{document}
